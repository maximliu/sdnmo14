\label{naas-sdn}

Having the concpet and architecture of OpenNaaS presented, in this section we
focuse on enhancements that highlight OpenNaaS serivce withSDN technology and
corresponding functional requirements are derived from those enhancements. 

\subsection{Enhancements}
	
	\subsubsection{Programmable Virtual/Physical Network Appliances}

	The separation of control and data-forwarding mechanisms allows more flexible control
	of packet-forwarding devices. Comparing to classical way of managing
	network devices, behaviour of SDN-compatible devices can be programmed
	through their north-bound interfaces on centralised management station. A
	SDN switch, for example, can be easily programmed to behaviour not only
	packet forwarding device, but also as a firewall or a load-balancer. The
	instructions can be realized in just few lines of code. 

	For OpenNaaS framework, a set of SDN-compatible devices can be programmed
	into different network appliances \emph{according to usage scenarios}.  In
	this way, programmable network can not only be reused but also aggregated
	or split in an on-the-fly manner which provide flexibility from both user
	and administrator perspective of views. Additionally middle-boxes could be
	replaced by unified, programmable and multi-purposed devices, which
	simplifies management and maintenance as well. 

	
	\subsubsection{Refined NaaS Control of Network Resources}

	Management rules and policies can be explicitly stated and enforced using
	proper programming or configuration languages, e.g. \cite{maple, frenetic}.
	Rules can be inserted into device's flow table where forwarding mechanism
	behaves accordingly, for example if data packets with certain origins have
	to traverse through a pre-determined switches can be expressed as flow
	rules.

	In terms of OpenNaaS, a refined control of network resources can
	dramatically reduce the overhead for administrator to configure networks
	with complex user requirement. Policies can be singly used or even be
	negotiated with user to adapt to their needs.

	\subsubsection{Incorporating Advanced Analytics of Networks Behaviors}

	SDN technology inherently facilitates in-network information gathering and
	analysis in order to gain deeper insights of network behaviours and event
	in real-time. Current OpenFlow specification~\cite{of14} for example,
	allows asynchronous and symmetric messages to be sends from devices to
	controller which contain network events such as \texttt{Packet-in/out},
	\texttt{Port-status}, \texttt{Flow-Removed} etc. Given availability of such
	types of messages provided by the resource layer devices, complex analytics
	can be done in order to assist either human operators or automated
	mechanisms to understand network behaviours and reacts to network event if necessary. 
	Analytics can be done either at controller layers at integral part of network OS 
	or controller can simply make those data available through its northbound interface 
	for the above layer to conduct analysis. Note that depending on the scenario, deluge of 
	data may be required to be processed in a real-time in order to achieve the goal, thus
	analytics in network is also an across research area with big data.

	
	\subsubsection{Incorporating QoS into on-demand Network}

	With support of configuration or programming languages, QoS requirements of
	network services created by NaaS can be expressed explicitly as rules in
	flow tables so that QoS policies are reflected and realized, for example
	the control of bandwidth for NaaS virtual networks with different quality
	and load-balancing requirements could be enforced using SDN rules expressed
	in the form of functions \cite{pyretic} as in conventional programming
	languages. 
	

	\subsubsection{Enabling Multi-domain SDN/OpenNaaS Service}
	
	Networking capabilities represented by both virtual and physical devices or
	resource slices can be orchestrated to form complex networking services.
	Since state-of-the-art SDN specification does not support inter-domain
	networking, using other inter-domain capable resources allows SDN networks
	in a single domain to forward their data packet to other domains, for
	instance, if particular kind of traffic must be routed outside of its own
	domain, an underlying SDN switches can be programmed with inter-domain
	routing capabilities such as EGP or BGP, for example a SDN-based inter-domain
	data exchange component at the edges of an organization can be a potential solution
	to allow multi-domain NaaS service. 
	
	\subsubsection{Allowing Continuum in the Network Management}

	Continuum in the network management refers to implementation and refinement
	of management policies or rules from high-level perspective to low-level
	executable management actions. Due to lack of an unified method, providing
	a consistent network management continuum is currently difficult, if at all
	possible. For services such as NaaS, a flexible control has to be in place
	in order to allow administrators to re-configure the network resources to
	adapt to ever changing management policies. A set of executable actions
	down to network interfaces level must be resulted to reflect the change of
	policy changes. With SDN policy languages such as Pyretic~\cite{pyretic}, a
	solid foundation could be built to express network management intentions
	in terms of a unified programming scripts in a generic way, programming
	language concepts such as functions and classes can be used to express the
	management rules and policies.


	\subsubsection{Intelligent Network Management with SDN}

	As underlying technology for NaaS platform, SDN works as a flexible
	intermediate layer between high-level services offered by NaaS and
	low-level infrastructures, e.g. switching/routing hardwares.  In a highly
	utilized NaaS environment, SDN devices may generate a large number of
	events and messages from OpenFlow channels, for example, asynchronous and
	symmetric messages sent back and forth between controller and data layers.
	Those message carries important management information that reflects the
	current behavior, such as changes, faults such as mis-configurations and
	even security informations. Thus aggregation, correlation and in-depth
	comprehension of such messages is of vital importance for understanding the
	performance of network in a real-time manner. It is also a great assistant
	for activities such as debugging of network failures or finding bottle neck
	of networks.  Obviously this is not a trivial task, sheer mount of data and
	complex inter-correlation of data require the processing algorithms to be
	capable of dealing with such data deluge in a highly efficient manner.
	Methods available to data-mining and machine learning researches can be
	explored for this purpose.
	
	\subsubsection{Coherent Resilience and High Availablity}
	
	


	\subsection{Requirements}
