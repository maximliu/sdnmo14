\label{introduction}

Flexible network infrastructures are crucial part for today's IT landscape.
Advance in hardware technologies constantly improves performance of
networking devices to suite the needs of applications and accommodate
simultaneous traffic flows. On the other hand, usage scenarios of today's
network infrastructures are very dynamic,ranging from networking for cloud
computing and Network Functions Virtualization (NFV) to deal with short-lived
network traffic spikes such as flash-crowd effect, providers should react agilely to
frequently changing user requirement on networks. This requires more flexibilities
in configuration as well as provisioning of resources in a very short time due
to limited reaction time. 

Considering those two aspects, an ultimate challenge currently faced by network
service providers today is thus how to deliver dynamic network infrastructures
at relatively low costs and in a timely manner. Furthermore network element should be 
application-aware as well, which means requirements from high level applications should 
be considered and reflected in the underlying network infrastructure.
%Such flexible and on-demand
%infrastructures are also required to address individual needs depend on usage
%scenario set by users and their specific applications. 
Obviously traditional network paradigm which purely based on dedicated
networking hardware is not efficient to address this challenge due to
its lacking of flexibility and reconfigurability. Recently a paradigm called
Network-as-a-Service (NaaS)~\cite{naas} which based on network virtualization
technology, shreds lights on this problem. With a layer of software entity,
NaaS paradigm allows creation of virtualized switches, routers and optical
devices ect. using physical devices, thus enables multiple tenants to use a
slice of shared resources according to their specific needs. NaaS can be
seamlessly integrated into other virtual services such as cloud computing and
NFV. NaaS allows more control of network elements from users' perspective of
view. Despite its conceptual advantages, it is still an unsolved question that
how to efficiently control physical network resources in a fine-granulate and precise 
way. 
%
%In order to allow National Research and Education Networks (NREN) and other 
%e-Infrastructure providers across Europe to collaborate and support research communities
%with highly flexible network infrastructures, a framework based on the paradigm of 
%NaaS is created by FP7 Mantychore project. The framework is called OpenNaaS~\cite{opennaas}.
%

A recent effort towards programmable networks with Software-defined networking
(SDN) has attract vast research interests. The idea of separation of data
forwarding and control plane of networking devices can improve controllability
and manageability of networks drastically. If properly integrated, we believe
such flexibility can extent the management and usage of NaaS thus provides as a
plausible solution for sustain the network controls of physical network
devices.  In this paper, we analyze the advantages of integration of SDN into
NaaS paradigm and how such integration can be properly designed and
implemented. The main contribution of this work are:

\begin{itemize}
	\item Proposing an architecture which integrates SDN into NaaS;
	\item Sustaining and enhancing functionalities of NaaS with refined network controls;
	\item Convergence of network infrastructures with cloud computing and NFV.
\end{itemize}

This paper is organised as followings: Section \ref{background}
provides briefly background information regarding NaaS and its use cases;
in Section \ref{naas-sdn} we discuss in-depth on enhancements of OpenNaaS with
support of SDN technology and in the meanwhile provides detailed design of
integration; Section \ref{architecture} and  Section \ref{impl} both detailed
discussion on architecture and a prototypical simulation result is presented;
this paper is concluded by discussion on future works. 
